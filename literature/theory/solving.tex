\documentclass{article}

% Language setting
% Replace `english' with e.g. `spanish' to change the document language
\usepackage[english]{babel}

% Set page size and margins
% Replace `letterpaper' with `a4paper' for UK/EU standard size
\usepackage[letterpaper,top=2cm,bottom=2cm,left=3cm,right=3cm,marginparwidth=1.75cm]{geometry}


% Useful packages
\usepackage{amsmath}
\usepackage{graphicx}
\usepackage[colorlinks=true, allcolors=blue]{hyperref}
\usepackage{tikz}
\usetikzlibrary{shapes.arrows, positioning, arrows.meta}


\title{Finding the optimal path}
\author{Riccardo Dal Cero}

\begin{document}
\maketitle

Given that the evolution of the net worth is the following:
\[e_{t+1} = Z(\theta +\epsilon)k_t^\alpha + (1-\delta )k_t - (1+r)(c+k_t-e_t)\]
The value of the firm at time t is given by:
\[V_t(e_t) =\max_{k_t}  e_t + \beta V_{t+1}(e_{t+1})\]

Solving the belman equation, the FOC are given by:
\[\frac{\vartheta V_t(e_t)}{\vartheta k_t } =\frac{\vartheta e_t}{\vartheta k_t } +
 \beta \frac{\vartheta V_{t+1}(e_{t+1})}{\vartheta k_t } \]
\[\frac{\vartheta e_t}{\vartheta k_t} = 0\]
Using the envelope theorem:
\[\frac{\vartheta
V_{t+1}(e_{t+1})}{\vartheta k_t } = \frac{\vartheta
V_{t+1}(e_{t+1})}{\vartheta e_{t+1} }\frac{\vartheta e_{t+1}}{\vartheta k_t}\]
\[\frac{\vartheta V_{t+1}(e_{t+1})}{\vartheta e_{t+1} } = \beta(1+r)\]
\[\frac{\vartheta e_{t+1}}{\vartheta k_t} = Z(\theta +\epsilon)\alpha k_t^{1-\alpha}
-(\delta + r)\]
Thus:
\[\frac{\vartheta V_{t+1}(e_{t+1})}{\vartheta k_t } = 
\beta(1+r) Z(\theta +\epsilon)\alpha k_t^{1-\alpha}-(\delta + r) \]
So the FOC became:
\[\frac{\vartheta V_t(e_t)}{\vartheta k_t } = 0 + \beta [ k_t^{1-\alpha}-(\delta + r)] = 0 \]
The optimal level of capital at time t is:
\[\widehat{k} _t= {\frac{\delta + r}{\beta(1+r) Z(\theta +\epsilon)\alpha}}^{\frac{1}{1-\alpha}} \]
Plotting the graph in (assuming \(X=Z(\theta +\epsilon)\)) and \(y=\widehat{k} _t\)
\[ y={\frac{0.03+0.1}{0.8 x}}^{\frac{1}{1-0.8}}\]

\begin{figure}
    \centering
    \includegraphics[scale = 0.4]{K.png}
    \caption{\(X=Z(\theta +\epsilon)\) and \(y=\widehat{k} _t\) }
    \label{plot:k_static}
\end{figure}
Its interesting to see that if there is an increase in productivity the firm need less optimal capital K, while the most
low productivity firms need more capital to operate 
\end{document}