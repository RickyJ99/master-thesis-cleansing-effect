\documentclass{article}

% Language setting
% Replace `english' with e.g. `spanish' to change the document language
\usepackage[english]{babel}

% Set page size and margins
% Replace `letterpaper' with `a4paper' for UK/EU standard size
\usepackage[letterpaper,top=2cm,bottom=2cm,left=3cm,right=3cm,marginparwidth=1.75cm]{geometry}


% Useful packages
\usepackage{amsmath}
\usepackage{graphicx}
\usepackage[colorlinks=true, allcolors=blue]{hyperref}
\usepackage{tikz}
\usetikzlibrary{shapes.arrows, positioning, arrows.meta}


\title{Finding the optimal path}
\author{Riccardo Dal Cero}

\begin{document}
\maketitle
\section{Frictionless economy}

Given that the evolution of the net worth is the following:
\[e_{t+1} = Z(\theta +\epsilon)k_t^\alpha + (1-\delta )k_t - (1+r)(c+k_t-e_t)\]
The value of the firm at time t is given by:
\[V_t(e_t) =\max_{k_t}  e_t + \beta V_{t+1}(e_{t+1})\]

Solving the belman equation, the FOC are given by:
\[\frac{\vartheta V_t(e_t)}{\vartheta k_t } =\frac{\vartheta e_t}{\vartheta k_t } +
 \beta \frac{\vartheta V_{t+1}(e_{t+1})}{\vartheta k_t } \]
Strong assumption a change in \(k_t\) does not have an impact on \(e_t\) only on \(e_{t+1}\):
\[\frac{\vartheta e_t}{\vartheta k_t} = 0\]
Using the envelope theorem:
\[\frac{\vartheta
V_{t+1}(e_{t+1})}{\vartheta k_t } = \frac{\vartheta
V_{t+1}(e_{t+1})}{\vartheta e_{t+1} }\frac{\vartheta e_{t+1}}{\vartheta k_t}\]
\[\frac{\vartheta V_{t+1}(e_{t+1})}{\vartheta e_{t+1} } =1 + \beta(1+r)\]
\[\frac{\vartheta e_{t+1}}{\vartheta k_t} = Z(\theta +\epsilon)\alpha k_t^{\alpha-1}
-(\delta + r)\]
Thus:
\[\frac{\vartheta V_{t+1}(e_{t+1})}{\vartheta k_t } = 
[1+\beta(1+r)] [Z(\theta + \epsilon)\alpha k_t^{\alpha-1}-(\delta + r)] \]
So the FOC became:
\[\frac{\vartheta V_t(e_t)}{\vartheta k_t } = 0 + \beta [1+\beta(1+r)] [Z(\theta + \epsilon)\alpha
k_t^{\alpha-1}-(\delta + r)]  = 0 \]
The optimal level of capital at time t is:
\[\widehat{k} _t= {\frac{\beta(1+r) Z(\theta +\epsilon)\alpha}{\delta + r}}^{\frac{1}{1-\alpha}} \]
Plotting the graph in (assuming \(X=Z(\theta +\epsilon)\)) and \(y=\widehat{k} _t\)
\[ y={\frac{0.8 x}{0.03+0.1}}^{\frac{1}{1-0.8}}\]

\begin{figure}
    \centering
    \includegraphics[scale = 0.8]{K.png}
    \caption{\(X=Z(\theta +\epsilon)\) and \(y=\widehat{k} _t\) }
    \label{plot:k_static}
\end{figure}
Its interesting to see that if there is an increase in productivity the firm need more optimal capital K, while the most
low productivity firms need less capital to operate 


\subsection{Frictions}
Now I consider the participation constraint of the borrower given that she could observe \(\epsilon\) only paying \(\mu
k^\alpha\) so:
\[
 (1+r )(k+c+e)(1-\Phi(\overline{\epsilon}))+\int_{-\infty}^{\overline{\epsilon}}[Z(\theta+\overline{\epsilon})
k^\alpha+(1-\delta)k-\mu k^\alpha] \,d\Phi(\epsilon) \geq (1+r)(c+k+e)
 \]
\(r\) is the rate of interest that makes equal the expected value of borrowing to the opportunity cost of capital.
rewriting became 
\[
Z[\theta+G(\overline{\epsilon} )]k_t^\alpha+(1-\delta)k_t-uk_t^\alpha\Phi (\overline{\epsilon})=(1+r)(k_t+c-e_t)
\]
where
\[G(\overline{\epsilon} )= (1-\Phi(\varepsilon )\overline{\varepsilon }
+\int_{-\infty}^{\overline{\varepsilon}}\epsilon\,d \Phi(\epsilon) )\]
While the firm participation constraint is \(q_t \geq 0 \) so the end-of-period net worth must be greater than 0, thus
the problem of the firm becomes:
The value of the firm at time t is given by:
\[V_t(e_t) =\max_{k_t}  q_t - e_{t+1} + \beta V_{t+1}(e_{t+1})\]
\[s.t.\]
\[q_{t} = Z(\theta +\epsilon)k_t^\alpha + (1-\delta )k_t - (1+r)(c+k_t-e_t)\]
\[
Z[\theta+G(\overline{\epsilon} )]k_t^\alpha+(1-\delta)k_t-uk_t^\alpha\Phi (\overline{\epsilon})=(1+r)(k_t+c-e_t)
\]
so we can rewrite the second constraint in order to get \(r\):
\[r = \frac{Z[\theta+G(\overline{\epsilon} )]k_t^\alpha+(1-\delta)k_t-uk_t^\alpha\Phi (\overline{\epsilon})}{k_t+c-e_t}-1\]
FOC:
\[\frac{\vartheta
V_{t}(e_{t})}{\vartheta k_t } = \frac{\vartheta
q_{t}}{\vartheta k_t } - \frac{\vartheta
e_{t}}{\vartheta k_t } + \beta \frac{\vartheta
V_{t+1}(e_{t+1})}{\vartheta k_{t} } = 0\]
by envelope theorem:
\[\frac{\vartheta
V_{t+1}(e_{t+1})}{\vartheta k_{t} } = \frac{\vartheta
V_{t+1}(e_{t+1})}{\vartheta e_{t+1} }\frac{\vartheta e_{t+1}}{\vartheta k_t}\]
Strong HP: \[\frac{\vartheta e_{t+1}}{\vartheta k_t} = 0\]
Qui non so se ha senso continuare perchè non ho nessun meccanismo che mi trasformi il net worth al tempo t in net worth
al periodo t+1, almeno che non includa la definzione di dividendo come \(d_t= q_t-e_{t+1}\), allora in questo caso
dovrei risolvere il problema con un langrangiana dato che la firm dovrebbe scegliere sia \(k\) che \(e\) net worth.
\end{document}