
Two influential theoretical models analyze the cleansing effect of recessions on the economy. The first model,
introduced by \cite{CabHarm94}, utilizes a vintage model of creative destruction to investigate how industries adapt to
cyclical demand variations. This model highlights how recessions can facilitate the removal of outdated and less
efficient production units, potentially leading to an overall increase in industry productivity. The model's insights
are derived from a framework where production units of varying vintages coexist, and the rate of technological adoption
plays a crucial role in determining industry responses to economic fluctuations.
\par

In contrast, the second model explored in this review, by \cite{OsePap17}, introduces financial frictions into the
analysis of the cleansing effect of recessions. This addition brings a new layer of complexity to the understanding of
how economic downturns affect firm dynamics, particularly by influencing the selection process through which firms enter
and exit the market. The model underscores the role of credit constraints in mediating the impact of recessions,
suggesting that financial frictions can dampen the potential productivity gains that might otherwise arise from the
purging effects of a downturn.
\par

Both models rely on numerical methods to solve their respective frameworks, acknowledging the intricate dynamics and
non-linearities inherent in their analyses. By comparing these models, this review seeks to illuminate the diverse
mechanisms through which recessions can influence economic outcomes, as well as the varying implications of introducing
different types of market frictions into theoretical frameworks.

\par
In the first paper that rationalizes the cleansing effect of recessions, \cite{CabHarm94} investigate how industries
respond to cyclical variations in demand using a vintage model of creative destruction. The underlying concept
postulates that the processes of creation and destruction within an industry partially explain business cycles.
Industries continuously experiencing creative destruction can adapt to demand fluctuations by adjusting the rate at
which they produce new units embodying advanced techniques or by altering the rate at which outdated units are retired.
The model incorporates heterogeneous firms, where production units embody the most advanced technology at the time of
their creation. The costs associated with creating new units slow down technology adoption, resulting in the coexistence
of production units with varying vintages.
\par

Key to understanding how firms adapt to business cycles are the concepts of the creative margin and the destruction
margin. For example, a demand reduction can be accommodated either by reducing the rate of technology adoption or by
retiring older production units. One of the primary factors determining which margin is more responsive to business
cycles is the adjustment cost. When this cost follows a linear pattern, the study shows that insulation is complete, and
the industry's response relies exclusively on its creation margin. Consequently, the creation margin becomes smoother
over time in comparison to the destruction margin, which exhibits greater responsiveness to the business cycle.
\par

Crucially, \cite{CabHarm94}'s research offers theoretical insights supported by empirical evidence. Their findings on
the cyclical nature of the destruction margin align with the studies conducted by \cite{BlaDia90} and \cite{DavHalt92}.
This convergence between theoretical framework and empirical substantiation underscores the importance of comprehending
the dynamic interplay between creative destruction and business cycles, which significantly influences how industries
respond to economic fluctuations.
\par

In their study, \cite{DavHalt92} assess the heterogeneity of employment changes at the establishment level in the U.S.
manufacturing sector from 1972 to 1986. It is revealed that job destruction exhibits procyclical tendencies, responding
more robustly to downturns in the economic cycle compared to the creation rate, in line with the theoretical model
proposed by \cite{CabHarm94}. The authors leverage a natural experiment inherent in the data to examine whether the
structure of adjustment costs can account for the behavior of these two margins. This natural experiment arises from the
asymmetric nature of business cycles, with recessions being shorter but more severe than expansions. The theoretical
model predicts that these differences should be attenuated in the creation process, a prediction that is substantiated
by the data since creation exhibits relative symmetry around its mean, while destruction displays a high degree of
asymmetry.
\par

The underlying concept driving the behavior of the destruction margin can be traced back to the theories of Schumpeter
and Hayek. They suggest that recessions represent periods during which unprofitable or outdated techniques are pruned
from the economy, leaving behind the most efficient firms \cite{HaCa07}. \par
The second model, by \cite{OsePap17}, introduces financial frictions into the analysis of the cleansing effect of
recessions. This model highlights how credit constraints can influence firm dynamics during economic downturns.
Financial frictions, such as borrowing constraints and monitoring costs, can dampen the potential productivity gains
that might otherwise arise from the purging effects of a downturn. The model suggests that financial frictions can lead
to the premature exit of some high-productivity firms, thereby reducing the overall productivity gains from recessions.
The model incorporates a framework where firms face a choice between continuing operations and exiting the market based
on their productivity and financial health. Firms with higher leverage may face greater financial stress during
downturns, influencing their survival and contributing to the market's cleansing process. Similarly, variations in
productivity affect firms' ability to withstand economic shocks, further delineating the selection process during
recessions. The findings from this model suggest that while financial frictions reduce the cleansing effect of recessions, the
latter remains positively significant. This aligns with the empirical evidence provided by \cite{OsePap17}, which shows
that despite the dampening influence of credit constraints, recessions retain an inherent capacity to purge the economy
of less efficient firms.
\par
This thesis contributes to the literature by developing a theoretical framework that explores how financial frictions
affect firms' decisions on the optimal trajectory for capital and dividends. Initially, the model excludes the
possibility of debt financing for investment, akin to the Ramsey-Cass-Koopmans model, with dividends instead of
consumption levels. By employing a Lagrangian method, the Euler equation for dividends is derived. In the next stage,
firms are allowed to finance investments through debt and retained earnings. The model introduces asymmetric information
via monitoring costs included in the participation constraint of financial intermediaries. Additionally, a borrowing
constraint that sets a fixed leverage ratio for firms represents another financial friction. The solution process starts
with a Lagrangian approach to formulate the Euler equations for dividends, followed by a Bellman equation with a
guess-and-verify method to derive a closed-form solution for the optimal paths of dividends and capital. An important
finding is that monitoring costs lead to lower steady-state levels of capital and dividends. The thesis concludes with
Monte Carlo simulations that introduce heterogeneity across firms and continuous sinusoidal productivity shocks. These
simulations suggest that financial frictions reduce the cleansing effect, but the latter remains positively significant,
aligning with the findings of \cite{OsePap17}.
\par

By comparing these two models and introducing a new theoretical framework, this literature review highlights the diverse
mechanisms through which recessions can influence economic outcomes. The vintage model of creative destruction by
\cite{CabHarm94} emphasizes the role of technological adoption and the coexistence of production units with varying
vintages, while the model by \cite{OsePap17} underscores the impact of financial frictions on firm dynamics. Together,
these models, along with the contributions of this thesis, provide a comprehensive understanding of the cleansing effect
of recessions and the varying implications of introducing different types of market frictions into theoretical
frameworks.
