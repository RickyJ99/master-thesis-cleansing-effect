\chapter{Conclusions}

% Recessions have always been time where the economy experienced higher rate of exit for firms, and higher reallocation
% rate for the labour rate as stated by \cite{DavHalt92}, and this was always associated with a following periods in which
% average productivity increases. However in the Great recession, the US experienced a lower reallocation rate since the
% creation rate that remeined lower than the previus recessions. A paper of \cite{FosHal16} found that reallocation
% patterns founded in previuse recessions are reversed during the great recession. \cite{OsePap17} reviseted the cleansing
% effect main theoretical model present in the literature \cite{CabHarm94}, introducing financial frictions, in order to
% understanfd if such frictions are responsible for the change in the reallocation behaviuor of firms. 
In the usual course of recessions, as \cite{DavHalt92} noted, there's typically an uptick in the number of firms
shutting down and a boost in the movement of workers—factors that are often followed by an increase in average
productivity. However, the Great Recession deviated from this pattern in the U.S., with a sluggish reallocation rate and
a persistently low rate of new firm creation, unlike what was seen in prior downturns. The research from \cite{FosHal16}
points out that the reallocation trends that were common in past recessions didn't hold up during the Great Recession.
In this context, \cite{OsePap17} took a fresh look at the key theoretical model of the cleansing effect from the
existing literature, specifically the model by \cite{CabHarm94}, by incorporating the concept of financial frictions to
probe whether these frictions were at play in altering the reallocation behavior of firms during such challenging
economic times. 
% 
This thesis revisits the established model by incorporating borrowing constraints that limit the amount of debt a firm
can incur. Through this adjustment of the theoretical framework, it has been possible to derive a closed-form solution
for the optimal trajectories of dividends and capital. This solution enables a direct examination of how monitoring
costs and a predetermined leverage ratio influence steady-state capital. 

The findings from the model reveal that monitoring costs, which financial intermediaries face when a firm defaults,
reduce the optimal trajectory and the steady state of capital and dividends. This makes firms more susceptible to the
adverse effects of economic downturns. Notably, this impact is more pronounced for highly productive firms, which feel a
greater effect on their steady-state capital and dividends. The reason behind this is linked to the assets that can be
collateralized in this model, represented by the ratio of output to debt. More productive firms, capable of generating
more output with the same amount of capital, would typically benefit from lower interest rates. However, because
monitoring costs scale with output, they diminish the positive impact that productivity has on interest rates. In
scenarios where monitoring costs match the output-to-debt ratio, the advantage of higher productivity vanishes entirely. 
% Another Interestingly finding is regarding leverage, the effect of leverage to the steady state capital is not trivial
% since the partial derivatives depend on level of leverage: for leverage above a certaint level the derivative is
% negative indicating above a certaint threshold the derivatives became positive. Thus if the borrowing constrint is tight
% firms can benefit from a lowering the leverage. So in an economy in which regulation does not allows the firms to use
% leverage over a certain amount incurs in the risk of making the firm works below optimal capital and by adjusting such
% constraint upward would make the firm better off, uneder certain conditions.
Another noteworthy result pertains to the influence of leverage on steady-state capital. This effect is ambiguous, as the
partial derivatives are sensitive to the actual leverage level: the derivative becomes negative when leverage surpasses
a specific threshold, indicating that excessive leverage reduces steady-state capital. Conversely, within the bounds of
stringent borrowing constraints, thus below the leverage level that makes derivative positive, firms might find advantages in increasing their leverage. Consequently, in regulatory
environments that cap leverage at a low level, there's a potential risk of firms operating with suboptimal capital.
Relaxing these constraints to permit slightly higher leverage could, under certain conditions, enhance the firm's
financial health and stability. 
% It is important to notice that these two finacial frictions are not indipendent one another,  and their effect add upp
% even if those effect can be of opposite sine as in the case of show above in which leverage is so low that an increase
% in leverage can increase steady state capital and thus production. However, is more realistic thinking to the case in
% which the leverage is above the low threshold that was making the patial derivative positive. Thinking about great
% recession if we consider an increasing in the leverage constraint and an increase in the monitoring cost those two
% effects add up, leading to a relevant reduction in steasy state capital and thus output and dividends. 
It's crucial to understand that the financial frictions of borrowing constraints and monitoring costs are
interconnected, and their impacts are cumulative, sometimes even counteracting each other. For instance, as
demonstrated, when leverage is initially low, an increase in leverage can actually boost steady-state capital and
production. Yet, it's more common and practical to consider situations where leverage exceeds this minimal threshold,
reversing the positive effect on the partial derivative. Reflecting on the Great Recession, if we look at tighter
leverage restrictions combined with heightened monitoring costs, these two factors together contribute to a significant
decrease in steady-state capital, as well as in overall output and dividends. 
% This can
% can give a hint why the Great Recession was different from the other recession in the reallocation pattern, as suggested
% by \cite{FosHal16}. Indeed, with montecarlo simulations was possible to simulate an economy which experienced a
% continous sinosoidal shock in productivity, with heterogeneous firms  by leverage and productivity. To represents the
% cleansing effect had been added an exit mechanism which force firm to exit if their return on capital belongs to the two
% lowers and if it is below the risk free rate. Interestingly, the economy subjected to monitoring costs reports a lower
% cleansing effect on avarage productivity, while the economy that was not subjected to monitorng costs found an higher
% cleansing effects. Those simulation want to emphasize that in presence of financial recession where the financial
% frictions are stronger the benefit of reallocation of capital due the exit of firms is lowered, this aligns with the
% findings of \cite{OsePap17}.
This observation may explain why the Great Recession differed from previous downturns in terms of firm reallocation
patterns, as \cite{FosHal16} have highlighted. Through Monte Carlo simulations, it was possible to model an economy
under continuous sinusoidal productivity shocks, featuring firms varied by leverage and productivity. An exit mechanism
was introduced to simulate the cleansing effect, forcing firms to exit the market if their return on capital was among
the lowest or fell below the risk-free rate. Notably, economies facing monitoring costs showed a diminished cleansing
effect on average productivity, in contrast to those without such costs, which experienced an enhanced effect. These
simulations underscore that during financial recessions with intensified financial frictions, the advantages of capital
reallocation through firm exits are reduced, corroborating the findings of \cite{OsePap17}. 
% In conclusion this theoretical model allowed to understand better how financial frictions such as borrowing contraint
% and monitoring costs can affect the optimal path and the steady starte for capital and dividends. While the montecarlo
% simulations give an hint to the macroeconomics effect that this can have. 
In conclusion, this theoretical model has shed some light on how financial frictions, like borrowing constraints and
monitoring costs, affect the paths and steady states for capital and dividends. The Monte Carlo simulations hint at why
financial crises might show different reallocation patterns than other recessions, providing a starting point for
understanding these complex dynamics.  