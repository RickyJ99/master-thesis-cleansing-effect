Recessions are characterized by increased instances of bankruptcies and business shutdowns. Notably, during the Great
Recession, the annual establishment exit rate in the US escalated from 11.8\% to 13.5\% from March 2008 to March 2009.
This surge in firm exit rates during economic contractions has propelled the notion that recessions serve a 'cleansing'
role in the economy: inefficient firms, becoming unprofitable, are phased out, thereby facilitating the redistribution
of resources to more productive entities. 
\par
This traditional perspective, highlighted in the work of \cite{CabHarm94}, presupposes that markets inherently favor the
survival of the most productive firms. Nonetheless, this premise has been contested by numerous studies indicating that
a firm's likelihood of exiting is influenced not just by its productivity but also by its ability to secure credit.
These insights imply that, amidst credit frictions, even highly productive firms that are financially strained might be
compelled to leave the market. Consequently, credit frictions could potentially undermine the productivity-boosting
impact of recessions as found in \cite{OsePap17}. 
\par
This thesis investigates the impact of credit frictions on the cleansing effect during recessions by examining firm
dynamics and credit constraints. Initially, it introduces a theoretical model to analyze how financial frictions
influence firm decisions on capital, debt, and dividends. Monte Carlo simulations are then applied to assess potential
implications for the cleansing effect of recessions. 
\par
The model is divided into two scenarios: in the first, investment is funded solely through retained earnings; in the
second, firms also have the option to finance through debt. Without the ability to borrow, the model aligns with
Ramsey-Cass-Koopmans frameworks, substituting family savings with firm savings and consumption with dividends. Firms
face capital depreciation and can only replace it by retaining dividends, depicted in the funds flow constraint. Using a
Lagrangian function, the Euler equation for dividends is derived, leading to steady-state conditions for dividends and
capital, visualized in a phase diagram. 
\par
The second part introduces financial frictions and debt financing for investments, aiming to find closed-form solutions
for the optimal paths of dividends and capital. Credit restrictions are shaped by asymmetric information and monitoring
costs. By incorporating a one-period financial contract, akin to the approach by \cite{BerGer86}, into the firm dynamics
model, it outlines borrowing constraints and interest rates set by financial intermediaries based on each firm's
productivity and leverage. The model also applies a borrowing constraint, fixing leverage over time. 
\par
The novelty introduced in this thesis to the literature is the development of a closed-form solution for the optimal
trajectories of dividends and capital under financial frictions. This is achieved using a Bellman equation combined with
a guess-and-verify approach, employing a logarithmic utility function. The derived solution demonstrates the significant
impact of monitoring costs, which reduce both capital and dividends, thereby increasing the vulnerability of firms
irrespective of their productivity. 
\par
Following the theoretical developments, this thesis employs Monte Carlo simulations to analyze the behavior of firms
differentiated by their productivity and leverage ratios in the face of business cycle fluctuations. These fluctuations
are modeled as continuous sinusoidal shocks to the productivity component, simulating the economic cycles. A key feature
of the model is the exit mechanism, which forces firms to exit the market if their return on capital falls below the
risk-free rate, directly engaging with the discourse on the cleansing effect of recessions.  
\par
The heterogeneity in firm characteristics, in terms of productivity and leverage, allows for an examination of how
heterogeneous firms respond to economic cycles and credit constraints. Firms with higher leverage may face greater
financial stress during downturns, influencing their survival and contributing to the market's cleansing process.
Similarly, variations in productivity affect firms' ability to withstand economic shocks, further delineating the
selection process during recessions. 

The simulations suggest that recessions enhance the economy's overall output by reallocating resources towards more
efficient firms, particularly in scenarios without monitoring costs. However, the introduction of such costs lowers this
efficiency gain, underscoring the role that financial frictions play in influencing the economy's resilience during
downturns. This observed cleansing effect, albeit modulated by the presence of financial frictions, aligns with the
findings in \cite{OsePap17}. Despite the dampening influence of credit constraints, I found that recessions retain an
inherent capacity to purge the economy of less efficient firms. 