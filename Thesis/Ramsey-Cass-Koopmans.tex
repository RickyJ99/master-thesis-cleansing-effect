\section{Ramsey-Cass-Koopmans reinterpreted}
This section outlines the intertemporal maximization problem faced by the firm in the absence of debt, which is a
Ramsey-Cass-Koopmans revisited model, where there is a firm that seeks to maximize the utility of consumption of the
shareholders (= dividends).
The objective function is:
\[V_0 = \sum_{t=0}^{+\infty}{\beta^t U(d_t)},\]
where \(U^{\prime}>0, U^{\prime \prime}<0\).
Let's assume that the firm's investment is entirely financed by equity (\(b_t=0\)  for all \(t\)), this leads to a simplified flow-of-funds constraint equation:
\begin{align}
    k_{t+1} = k_{t}(1 - \delta) + f(k_{t}) - d_{t}.  \label{eq13}
\end{align}
The maximization problem is tackled using a Lagrangian method, where the Lagrangian is defined as:
\[L_0 = \sum_{t=0}^{+\infty}\left[{\beta^t U(d_t) - \beta^t \lambda_t\left[k_{t+1} - k_{t}(1 - \delta) - f(k_t) + d_t\right]}\right].\]

The first-order conditions for \(d_{t}\), \(k_{t}\), for all periods \(t=0,1,\ldots\) yield:
\[
U^{\prime}(d_{t}) = \lambda_t, \quad \forall t,
\]
\[
\beta^t \lambda_t = \beta^{t+1} \lambda_{t+1}[f^{\prime}(k_{t+1}) + (1-\delta)], \quad \forall t,
\]


This approach delineates the optimal strategy for dividend distribution and capital allocation in a debt-free
case.
In the infinite horizon model, the transversality condition reads:
\begin{align}
    \lim _{t \rightarrow \infty} \beta^T U^{\prime}\left(d_{t}\right) k_{T+1}=0 \label{eq26}
\end{align}
From these first-order conditions (FOCs), we derive the Euler equation for dividends:
\begin{align}
    U^{\prime}(d_{t}) = \beta U^{\prime}(d_{t+1})[f^{\prime}(k_{t+1}) + (1-\delta)]  \label{eq14}
\end{align}
indicating that the marginal utility of distributing 1 unit of output as dividends at time \(t\) should match the discounted marginal
utility of not distributing  dividends in t, saving, investing in t, using the additional capital to produce and
distribute the corresponding dividends at time t+1.
\paragraph{Steady state condition for dividends}
Imposing the steady state condition for dividends \( d_t = d_{t+1} = \widehat{d} \) in \ref{eq14} , we
equate the marginal  utilities across two consecutive periods:
\begin{align}
    U^{\prime}(d_t) &= U^{\prime}(d_{t+1}), \nonumber \\
    \frac{1}{\beta} = f^{\prime}(k_{t+1}) + (1-\delta),\label{eq27}
\end{align}
This condition is satisfied if:
\begin{align}
    f^{\prime}(k_{t+1}) &= \frac{1}{\beta} - (1-\delta), 
\end{align}
Using the Cobb-Douglas production function and taking the derivative with respect to capital, we get:
\begin{align}
    f^{\prime}(k_{t+1}) &= Z \alpha k^{\alpha-1}_{t+1},  \label{eq15'}
\end{align}
From \ref{eq27}, we get: 
\begin{align}
    \frac{1}{\beta}&=1-\delta+Z\alpha k^{\alpha-1},\nonumber \\
    Z\alpha k^{\alpha-1} &+(1-\delta)\beta=1, \nonumber \\
    \widehat{k} &= \left[\frac{\alpha \beta Z}{1 - \beta\left(1-\delta\right)}\right]^{\frac{1}{1-\alpha}}.  \label{eq16}
\end{align}
The locus of points on the (\(k_t,d_t\)) plane such that dividends are constant is therefore
\(k_t=\hat{k}\), whose representation in the vertical line in figure \ref{fig:ph_d_k_nod_1}.
\paragraph{Steady state condition for capital}
Imposing the steady state condition for capital (\(k_t = k_{t+1} = \widehat{k}\)) in the law of motion of capital \ref{eq13} we
get:
\begin{align}
    d_t & = f(k_t) - \delta k_t  \label{eq17}
\end{align}
This locus represents the set of points where capital stock remains constant over time. Hence, we can determine the
level of dividends that ensures both capital and dividends are maintained at steady-state levels. By incorporating the
steady-state level of capital  from Equation \ref{eq16} and the production function from Equation \ref{eq6} into the equilibrium
condition for capital from Equation \ref{eq17}, the steady-state level of dividends, can be
deduced:

\begin{align}
    \widehat{d} &= Z \left( \frac{\alpha \beta Z}{1 - \beta(1 - \delta)} \right)^{\frac{\alpha}{1 - \alpha}} - \delta \left( \frac{\alpha \beta Z}{1 - \beta(1 - \delta)} \right)^{\frac{1}{1 - \alpha}}
\end{align}



In this manner, we ascertain the steady-state levels for both capital and dividends.
\subsection{Phase diagram}

In this section, we will plot the phase diagram for capital and dividends exploiting steady-state conditions for capital and
dividends. 
\begin{figure}
    \centering
    \begin{tikzpicture}
        \begin{axis}[
            axis lines=middle, % sets the position of the axes
            xlabel=\(k_t\),
            ylabel=\(d_t\),
            ymin=0, xmin=0,
            xmax=5, ymax=5,
            ticks=none, % removes ticks on axes
            clip=false,
            every axis plot post/.style={thick}
        ]
        
        % Vertical line at k hat
        \draw  (axis cs:1.8,0) -- (axis cs:1.8,5);
        \node at (axis cs:2,5.5) {\(\Delta d_t = 0\)};
        \node at (axis cs:1.8,-0.5) {\(\hat{k}\)};
        
        % Arrows
        % Upwards arrows
        \draw [-{Latex[length=3mm]}] (axis cs:1,1) -- (axis cs:1,2);
    
        % Downwards arrows
        \draw [{Latex[length=3mm]}-] (axis cs:3,1) -- (axis cs:3,2);

        \end{axis}
    \end{tikzpicture}
    \caption{The phase diagram illustrates locus of points such that \(d_t=d_{t-1}\). The vertical line at \( \hat{k} \) represents the steady-state level of capital, where
    the rate of change in dividends \( \Delta d_t \) is zero. To the left of \( \hat{k} \), where capital is below its
    steady-state level, the firm increases dividend payments. Conversely, to the right of \( \hat{k} \), where capital
    exceeds the steady state, dividend payments decrease.}
    \label{fig:ph_d_k_nod_1}
\end{figure}

The \ref{fig:ph_d_k_nod_1} portrays the dynamics of dividends (\(d_t\)) in relation to the capital (\(k_t\)) of a firm, with a
particular focus on the behavior when capital is below or above the steady-state level,  denoted by \(\hat{k}\).

When capital is below the steady-state level (\(k_t < \hat{k}\)), thus on the left of the vertical line, for the
firm it is optimal to increase dividends over time (\(d_t<d_{t+1}\)) as represented by the arrow pointing upward. When instead
\(k_t > \hat{k}\), dividends must shrink over time (\(d_t>d_{t+1}\)).
Let's look at the locus in which capital is stationary \(\Delta k = 0 \) is given by the f-of-f constraint \ref{eq17}:
\begin{align}
    d_t & = f(k_t) - \delta k_t
\end{align}
In this case, as obtained in the above section, the locus in which capital is stationary becomes \ref{eq17}:
\begin{align}
    d_t &=Z k^{\alpha}_t-\delta k_t
\end{align}

This function starts at the origin since (\(f(0)=0\)), with a maximum in \(\underline{k}\) (defined as capital level
such that \(f^{\prime}(\underline{k})=\delta\)). The level of capital \(\underline{k}\) is:
\begin{align}
    \underline{k} &= \left[\frac{\alpha Z}{\delta}\right]^{\frac{1}{1-\alpha}}  \label{eq19}
\end{align}
It is easy to see that \(\hat{k}<\underline{k}\):
\begin{align*}
    \frac{\alpha \beta Z}{1-\beta+\beta \delta}&<\frac{\alpha Z}{\delta},\\
    \frac{\beta}{1-\beta+\beta \delta}&<\frac{1}{\delta},\\
    \beta \delta &<1-\beta + \beta \delta, \\
    1-\beta&>0 
\end{align*}
We denote \(\bar{k}\) the capital level such that \(d_t=0\), thus it's obtained by solving \(f(\bar{k}) -\delta
\bar{k} = 0\), using the Cobb-Douglas production function \ref{eq6}, we get:
\begin{align}
    Z \bar{k}^{\alpha} &= \delta \bar{k}, \nonumber\\
    \bar{k}&=\left[\frac{Z}{\delta}\right]^{\frac{1}{1-\alpha}}
\end{align}
It is easy to see that \(\underline{k}<\bar{k}\), since:
\begin{align}
    \left[\frac{Z}{\delta}\right]^{\frac{1}{1-\alpha}}&>\left[\frac{\alpha Z}{\delta}\right]^{\frac{1}{1-\alpha}},\nonumber\\
    \frac{Z}{\delta}&>\frac{\alpha Z}{\delta}, \nonumber \\
    1&>\alpha \nonumber 
\end{align}
Considering transitivity, the sequence for \(\underline{k}, \hat{k}, \bar{k}\) is:
\begin{align*}
    \hat{k} < \underline{k} < \bar{k}
\end{align*}
\begin{figure}
    \centering
    \begin{tikzpicture}
        \begin{axis}[
            axis lines=middle, % sets the position of the axes
            xlabel=\(k_t\),
            ylabel=\(d_t\),
            xmin=0, ymin=0,
            xmax=5, ymax=5,
            ticks=none, % removes ticks on axes
            clip=false,
            axis on top=true
        ]
        
        % Parabolic curve
        \addplot [
            domain=0:4, 
            samples=100, 
            thick,
        ]
        {-x^2 + 4*x};
        
        % Vertical line at k hat

        \node at (axis cs:3.5,3.5) {\(\Delta k_t = 0\)};

        
        % Arrows on the left side
        \draw [-{Stealth[bend]}]  (axis cs:0.5,1)--(axis cs:1,1);
        \draw [-{Stealth[bend]}]  (axis cs:0.8,4) -- (axis cs:0.5,4);
        
        % Arrows on the right side
        \draw [-{Stealth[bend]}]  (axis cs:2.5,1)--(axis cs:3,1);
        \draw [-{Stealth[bend]}]  (axis cs:4,4) -- (axis cs:3.5,4);
        \node at (axis cs:2,-0.2) {\(\underline{k}\)};
        \node at (axis cs:4,-0.2) {\(\bar{k}\)};

        %vertical line at maximum
        \draw [dashed] (axis cs:2,0) -- (axis cs:2,4);
        \end{axis}
    \end{tikzpicture}
    \caption{This phase diagram displays the set of points at which the capital stock \( k_t \) is
    unchanging from one period to the next (\( k_t = k_{t-1} \)). The capital level at \( \underline{k} \) represents the
    maximum sustainable dividend payout without affecting the capital stock. Arrows above the curve pointing leftward
    indicate a reduction in capital resulting from dividend levels that exceed the sustainable steady state, while
    arrows pointing to the right below the curve signify the accumulation of capital due to dividend levels that are
    below the steady state, leading to an increase in capital stock over time. 
    }
    \label{fig:ph_d_k_nod}
\end{figure}

For a given level of capital \(k_0 \in [0,\bar{k}]\), the corresponding dividends level that
guarantee the stationarity of capital is:
\begin{align*}
    d_0 &= f(k_0) -\delta k_0
\end{align*}
If the firm distributes more dividends than \(d_0\) the capital stock must decrease over time: since
dividends are too high the firm is consuming part of her capital. More precisely the firm is distributing more dividends than
\(d_0\), which guarantees that the difference between production, and dividends is exactly equal to the replacement of
depreciated capital.
This behavior is represented by the arrows above the curve pointing to the left. If the firm distributes fewer
dividends than \(d_0\), the opposite happens: the firm increases its capital since there is a positive net
investment. This behavior is represented by
the arrows below the curve pointing toward the right. 

\paragraph{Steady state for capital and dividends}
Plotting both loci we get a phase diagram that represents the condition for stationarity. 
\begin{figure}
    \centering
    \begin{tikzpicture}
        \begin{axis}[
            axis lines=middle, % sets the position of the axes
            xlabel=\(k_t\),
            ylabel=\(d_t\),
            xmin=0, ymin=0,
            xmax=5, ymax=5,
            ticks=none, % removes ticks on axes
            clip=false,
            axis on top=true
        ]
        
        % Parabolic curve
        \addplot [
            domain=0:4, 
            samples=100, 
            thick,
        ]
        {-x^2 + 4*x};
        
        % Arrows
       
        
        % Vertical line at k hat
        \draw [dashed] (axis cs:2,0) -- (axis cs:2,4);
        \node at (axis cs:2.5,4.5) {\(\Delta d_t = 0\)};
        \node at (axis cs:3.5,3.5) {\(\Delta k_t = 0\)};
        \node at (axis cs:4,-0.2) {\(\bar{k} \)};

        \draw  (axis cs:1.8,0) -- (axis cs:1.8,5);

        \node at (axis cs:1.8,-0.2) {\(\hat{k} \)};
        % Arrows on the left side
        \draw [-{Stealth[bend]}]  (axis cs:1.3,1)--(axis cs:1.8,1);
        \draw [-{Stealth[bend]}]  (axis cs:0.8,3) -- (axis cs:0.3,3);
        
        % Arrows on the right side
        \draw [-{Stealth[bend]}]  (axis cs:2.5,1)--(axis cs:3,1);
        \draw [-{Stealth[bend]}]  (axis cs:4,3) -- (axis cs:3.5,3);
        
        % Arrows
        % Upwards arrows
        \draw [-{Stealth[bend]}]  (axis cs:0.8,3) -- (axis cs:0.8,3.5);
        \draw [-{Stealth[bend]}]  (axis cs:1.3,1) -- (axis cs:1.3,1.5);
        \draw [-{Stealth[bend]}]  (axis cs:2.5,1) -- (axis cs:2.5,0.5);
        \draw [-{Stealth[bend]}]  (axis cs:4,3) -- (axis cs:4,2.5);
        
        % Vertical line at k hat
        \draw [dashed] (axis cs:1,0) -- (axis cs:1,5);
        \node at (axis cs:0.8,2) {A};
        \node at (axis cs:1,-0.2) {\(k_0\)};
        \draw [dashed] (axis cs:1,0) -- (axis cs:1,5);
        
        % Draw a green dot at coordinates (1.1,1.1)
        \draw [-{Stealth[bend]}, color=blue]  (axis cs:1,2) -- (axis cs:1.76,3.87);
        \node[draw, circle, fill=green, inner sep=2pt] at (axis cs:1.81,3.96) {};
        \node[draw, circle, fill=green, inner sep=2pt] at (axis cs:1,2) {};
        \node at (axis cs:1.7,4.2) {B};
        
        \node at (axis cs:2,-0.2) {\(\underline{k}\)};
        \end{axis}
    \end{tikzpicture}
    \caption{The phase diagram visualizes the relationship between dividends and capital. The vertical line marks
    where the dividend level remains constant, and the solid concave curve traces where the capital remains constant.
    Point B indicates the equilibrium where both dividends and capital are stationary, with the steady-state capital at
    \( \hat{k} \). This value of \( \hat{k} \) is notably lower than \( \underline{k} \), which is the capital level
    that would maximize dividends while maintaining a steady capital stock. \( \bar{k} \) represents the capital
    quantity at which the system reaches a stationary state for k with zero dividends. The arrows illustrate the dynamics of dividends and capital in case of the firm finding itself outside the stable
    loci. Point A is a possible starting point (\(k_0,d_0\)). 
    The blue arrows indicate a potential trajectory, known as a saddle path, leading towards saddle point B. 
    }
    \label{fig:ph_d_nodebt}
\end{figure}
Notice that there exits 3 steady states: one at the origin due to the assumption \(f(0)=0\), the point \((\bar{k};0)\),
and finally point B \(=(\hat{k},\hat{d})\). Point B was obtained by equations \ref{eq16} and \ref{eq17} and represented the point in
which dividends and capital are at a steady state, and both are strictly positive. This point is a saddle point.
The blue line depicts a possible saddle path towards B. Starting at A, the firm chooses exactly the dividend level that
leads to the stationary point B. This path not only fulfills the difference equations \ref{eq14} and \ref{eq17}, but
also, the transversality condition \ref{eq26}
Indeed as \(t \rightarrow \infty\), capital and dividends approach their steady-state level, which are both positive and
finite, thus the marginal utility of dividends at \(\hat{d}\) is also finite, hence \ref{eq26} is valid.

To conclude, this section has outlined the derivation of steady-state levels for capital and dividends, and these
conditions have been visually represented in a phase diagram. The upcoming section will undertake a similar analysis but
will incorporate debt and financial friction into the model. Additionally, it is important to note that in the steady
state, where output \(\hat{y}\) is given by \(f(\hat{k})\), the production level is exclusively influenced by the
productivity parameter \(Z\), signifying that financial elements do not alter the fundamental connection between the
production function and output. 
