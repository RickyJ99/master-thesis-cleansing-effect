%% Appendix law of motion of capital and debt

\subsection{Dynamics of capital}
In the
simplified scenario where a firm operates without incurring debt (\(b_t = 0\) for all \(t\)) and imposing constant and
positive dividends (\(d_t=d>0\)), the flow of funds constraint changes as follows:
\begin{align}
    I_t + d &= f(k_{t}) \\
    I_t &= S_t \label{eq7}
\end{align}
In the absence of debt, a firm's investment is solely financed through retained earnings, as
specified in equation \ref{eq7}. From \ref{eq7}, we derive the difference equation describing the evolution of capital:
\begin{align}
    \Delta{k_{t+1}} &= f(k_{t}) - d -\delta k_t  \label{eq24}\\
    \Delta{k_{t+1}} &=k_{t+1}-k_t \nonumber
\end{align}
Equation \ref{eq24} presented implies that variation in capital stock depends on the balance between retained earnings and
depreciated capital. Thus, for a positive increase in capital \(\Delta{k_{t+1}} > 0\), it is necessary for a company to
hold back earnings over what is required to offset depreciated capital:
\begin{align*}
    f(k_{t}) - d &> \delta k_t
\end{align*}
Similarly, the reverse scenario holds as well.
To determine the steady-state level of capital, we apply the previously used production function (see equation \ref{eq6}),
setting \(k_{t+1}=k_{t}=\hat{k}\) within the difference equation (refer to equation \ref{eq24}): 
\begin{align}
    Z \hat{k}^{\alpha} - d&=  \delta \hat{k}
\end{align}

The slope of the phase diagram is:
\begin{align}
    \frac{\partial k_{t+1}}{\partial k_{t}} &= (1-\delta) + f^{\prime}(k_{t}) = \\
     &= (1-\delta) + \alpha Z k_{t}^{\alpha-1} \label{eq8}
\end{align}
Analysis reveals two distinct scenarios based on the derivative of capital with respect to \(k_{t}\):
\begin{enumerate}
    \item \textbf{Exploding Path When Derivative Is Greater Than 1}: If the partial derivative with respect to \(k_{t}\) is greater than 1, it suggests that small deviations from the
    steady-state level (\(\widehat{k}\)) will lead to increasing divergence rather than convergence back to
    \(\widehat{k}\). This scenario can be characterized by potentially unstable dynamics. The condition is:
    \[
    (1-\delta) + \alpha Z k_{t}^{\alpha-1} > 1, \quad \alpha Z k_{t}^{\alpha-1} > \delta, \quad k_{t} > \left(\frac{\alpha Z}{\delta}\right)^{\frac{1}{1-\alpha}}
    \]

    \item \textbf{Convergence to Steady State}: Conversely, if the derivative is less than 1, this indicates
    that deviations from the steady-state level will
    diminish over time, leading to a stable equilibrium. This scenario reflects a stable path where the system tends to
    return to its steady state following a perturbation.   The condition is:
    \[
    (1-\delta) + \alpha Z k_{t}^{\alpha-1} < 1, \quad \alpha Z k_{t}^{\alpha-1} < \delta, \quad k_{t} < \left(\frac{\alpha Z}{\delta}\right)^{\frac{1}{1-\alpha}}
    \]

\end{enumerate}

The following phase diagram represent the case \(\frac{\partial{k_t}}{\partial k_{t-1}} > 1\), using the following
parameters: \(\delta =
0.1\), \(\alpha = 0.8\), \(Z = 0.5\), and \(d = 0.8\)
\begin{figure}
    \centering
    \begin{tikzpicture}
        \begin{axis}[
            axis lines=center,
            xlabel=\(k_{t-1}\),
            ylabel={\(k_{t}\)},
            extra y ticks={0},
            extra y ticks={-0.8},
            xmin=0,
            ymin=-0.8,
        ]
        % Below the black line is defined
        \addplot [
            domain=0:5, 
            samples=100, 
            color=black,
            dotted,
        ]
        {x};
        
        % Horizontal line at y=2.5
        \draw [dotted, color=black] (axis cs:0,2.5) -- (axis cs:2.5,2.5);
        
        % Vertical line at x=2.5
        \draw [dotted, color=black] (axis cs:2.5,0) -- (axis cs:2.5,2.5);
        
        % Draw a red dot at coordinates (2.5,2.5)
        \node[draw, circle, fill=red, inner sep=2pt] at (axis cs:2.5,2.5) {};
        
        \node at (axis cs:2.5,1.8) {\(\widehat{k}\)};

        % Here the blue curve is defined with arrows
        \addplot [
            domain=0:5, 
            samples=100, 
            color=blue,
            postaction={
                decorate,
                decoration={
                    markings,
                    mark=at position 0.33 with {\arrow{<}},
                    %mark=at position 0.5 with {\arrow{>}},
                    mark=at position 0.66 with {\arrow{>}},
                }
            }
        ]
        {0.5*x^0.8 - 0.1*x + x - 0.8};
        

        \end{axis}
    \end{tikzpicture}
    \caption{The phase diagram demonstrates the evolution of capital in a debt-free scenario, with the dynamics dictated by the specific
    parameters:  depreciation rate (\(\delta =0.1\)), capital's output elasticity (\(\alpha = 0.8\)), total factor
    productivity (\( Z= 0.5\)), and fixed dividends (\(d = 0.8\)). The blue trajectory line depicts the capital accumulation
    process, calculated with the formula \( k_{t+1} = 0.5 \cdot k_{t}^{0.8} - 0.1 \cdot k_{t} + k_{t} - 0.8 \), which
    captures the interplay between capital growth through production, the diminishing returns as capital increases
    (reflected by the concavity of the curve), and the outflow due to depreciation and dividends. The steady-state
    capital (\(\widehat{k}\)), marked by a red dot, indicates the level where the economy naturally gravitates over
    time. At this point, the firm's investment is precisely calibrated to replace depreciated capital and issue
    dividends, with no additional net investment. Notably, this equilibrium is a focal point of the system; below this
    level, capital accumulates, and above it, capital stock adjusts downward, converging back to this stable point. The
    diagram serves as a visual aid to comprehend capital dynamics within this economic framework and underscores the
    balancing act between production, depreciation, and dividend distribution in long-term capital management.
    }

\end{figure}

The graph distinctly demonstrates that when capital at time \(t\) is below the red dot, it means that
capital is less than the steady-state capital, leading to a diminishing trajectory in the firm's capital. Conversely, if
capital is above the steady-state level, indicated by \(\widehat{k}\), the firm is overcapitalized, and the
trajectory becomes explosive, with capital increasing without bound. 
\\
If a firm's capital is less than the steady state, the outflows—such as
the replacement of depreciated capital and dividends—are disproportionately high compared to its production. This dynamic will inevitably
cause the firm's capital to go down to zero. It's crucial to recognize that this path is predicated on the
assumption of constant dividends.
\\
Furthermore, the steeper the slope of the blue line, the higher the productivity factor \(Z\), signifying a reduced need
for capital.
\vspace{1cm}
\subsection{Dynamics of debts}
To examine the dynamics of debt, consider a scenario where capital remains constant \(k_t=k_{t+1}=\widehat{k}\), thus it is at the
steady-state level. From equation \ref{eq5'} we get the difference equation for debt:
\begin{align}
    \Delta{b_{t+1}} &= d - f(\widehat{k}) + \delta \widehat{k} + r  b_{t}  \label{eq10}
\end{align}

To derive the steady-state debt level, we'll look for the point where debt doesn't change from one period to the next,
which means  \(\Delta{b_{t+1}} = 0\). This occurs when:
\begin{align}
    d &- f(\widehat{k}) + \delta \widehat{k} + r b_{t} = 0 
\end{align}
Solving for the steady-state debt level \( \widehat{b} \), we set \( b_{t} = \widehat{b} \) and get:
\begin{align}
    d &- f(\widehat{k}) + \delta \widehat{k} + r \widehat{b} = 0 
\end{align}
 where \( f(\widehat{k}) \) is the output of the firm given the
steady-state  capital \( \widehat{k} \). Since \( f(k) \) follows a Cobb-Douglas production function \ref{eq6},  then:

\begin{align}
    d &- Z \widehat{k}^\alpha + \delta \widehat{k} + r \widehat{b} = 0
\end{align}

Isolating \( \widehat{b} \) to find the steady-state level of debt, we get:
\begin{align}
    r \widehat{b} &= Z \widehat{k}^\alpha - \delta \widehat{k} - d, \nonumber\\
    \widehat{b} &= \frac{Z \widehat{k}^\alpha - \delta \widehat{k} - d}{r}
\end{align}
This equation gives us the steady-state level of debt \( \widehat{b} \), assuming that the output of the firm is enough
to cover depreciation and dividends, with the remaining used to service debt. If output is insufficient, the
firm would need to borrow more, and the steady-state debt would be higher. If the output exceeds the depreciation and
dividends, the firm can pay down the debt, and the steady-state debt would be lower. 
Let's determine the condition for a stable path taking the partial derivatives with respect to \(b_{t-1}\):
\begin{align}
    \frac{\partial{b_t}}{\partial b_{t-1}} &= 1+r  \label{eq11} \\
\end{align}
Since \(r >0\), the partial derivative will always be greater than one, thus the slope of the
difference equation for debt will always be steeper than one. Adding a negative intercept due to positive dividends we
get that under those conditions there exists a steady state of debt. Moreover, if the debt is below the steady state, the
debt will shrink toward 0, while if the debt is over the steady state the dynamics of debt will explode toward
\(+\infty\).
\begin{figure}
    \centering
    \begin{tikzpicture}
        \begin{axis}[
            axis lines=left,
            xlabel=\(b_t\),
            ylabel={\(b_{t+1}\)},
            ymin=0,
            xmin=0,
        ]
        % Below the black line is defined
        \addplot [
            domain=0:5, 
            samples=100, 
            color=black,
            dotted,
        ]
        {x};
        % Horizontal line at y=2.5
        \draw [dotted, color=black] (axis cs:0,2.5) -- (axis cs:2.5,2.5);
        
        % Vertical line at x=2.5
        \draw [dotted, color=black] (axis cs:2.5,0) -- (axis cs:2.5,2.5);
        
        % Draw a red dot at coordinates (2.5,2.5)
        \node[draw, circle, fill=red, inner sep=2pt] at (axis cs:2.5,2.5) {};
        
        % Horizontal line at y=2.5
        \draw [dotted, color=black] (axis cs:0,1.1) -- (axis cs:1.1,1.1);
        
        % Vertical line at x=2.5
        \draw [dotted, color=black] (axis cs:1.1,0) -- (axis cs:1.1,1.1);
        
        
        
        % Here the blue curve is defined with arrows
        \addplot [
            domain=0:5, 
            samples=100, 
            color=blue,
            postaction={
                decorate,
                decoration={
                    markings,
                    mark=at position 0.1 with {\arrow{<}},
                    %mark=at position 0.5 with {\arrow{>}},
                    mark=at position 0.3 with {\arrow{>}},
                }
            }
        ]
        {-0.5*3^0.8 + 0.1*3 + 1.1*x + 0.8};
        % Draw a red dot at coordinates (2.5,2.5)
        \node[draw, circle, fill=green, inner sep=2pt] at (axis cs:1.1,1.1) {};

        \end{axis}
    \end{tikzpicture}
    \caption{The phase diagram dynamically visualizes the firm's debt trajectory given a static capital stock ($\Delta k = 0$),
    operationalized within a model characterized by the parameters $\delta = 0.1$ (depreciation rate), $r = 0.1$
    (interest rate), $\alpha = 0.8$ (capital output elasticity), $Z = 0.5$ (total factor productivity), $d = 0.8$
    (constant dividend payout), and a steady-state capital ($\widehat{k} = 3$). The blue line embodies the trajectory of
    debt, guided by  the finite difference equation $b_{t+1} = (1+r) \cdot b_{t} - (f(\widehat{k}) -
    \delta \widehat{k} - d)$, where $f(\widehat{k})$ denotes the firm's output at the steady state of capital. This
    equation encapsulates the interplay between the interest on existing debt and the firm's obligations due to
    dividends and depreciation. The red dot marks the threshold beyond which debt cannot exceed capital, effectively serving
    as a limit 
    on debt. The green dot
    represents the equilibrium or steady-state debt level, where the firm's financial obligations are perfectly balanced
    with its repayment capacity. The distance between the red and green dots illustrates the magnitude of the firm's
    equity buffer, serving as a measure of financial health and resilience against market fluctuations. This diagram is
    pivotal in understanding how dividend policies and productivity rates interlink to shape the firm's leverage
    strategy and long-term financial stability.}
    \label{fig:ph_k}
\end{figure}
The phase diagram \ref{fig:ph_k} illustrates the relationship between a firm's current debt (\(b_t\)) and its capacity for future operations
(\(k_{t+1}\)), within the context of constant dividends. The steady state is indicated by the red dot, signifying the
juncture at which the firm's output is precisely adequate to cover dividends, depreciation, and interest on its
steady-state debt. %r
In essence, the graph conveys how steady-state conditions are shaped by dividend policy and productivity, with the
former influencing the firm's financial leverage and the latter determining its capital efficiency. 